\documentclass[a4paper,10pt]{article}
\usepackage[utf8]{inputenc}

%opening
\title{}
\author{}

\begin{document}

\maketitle

\begin{abstract}
otvorit server u browseru nek ceka za kasnije
\end{abstract}

\section{slajd}
Dobar dan, mi smo bla i ble. Danas \' cemo Vam predstaviti na\v s seminarski rad koji je usporedba nekoliko algoritama za tra\v zenje lokalnih ekstrema funkcija.

\section{slajd}
Ovo je struktura kojom \' cemo probat pobli\v ze objasniti sto smo i kako radili, te na kraju i demonstrirati na\v su aplikaciju koja pru\v za usporedbu tih algoritama.

\section{slajd}
\emph{pi\v se isto na slajdu, to poku\v samo zapamtit a ne da sve \v citamo ba\v s sa slajdova :)}
Dvije osnovne vrste problema:
  \begin{itemize}
   \item tra\v zenje optimuma na segmentu domene (npr. Schafferova F6 funkcija na $[-100,100]^2$)
   \item tra\v zenje optimuma funkcija ograni\v cenih sa nekim jednakostima i/ili nejednakostima na cijeloj domeni (npr. G funkcije).
  \end{itemize}

\section{slajd}
Dane problme rje\v savamo metaheuristikama. One se dijle na sljede\' ci na\v cin: \emph{(pokazat na slide i pro\v citat \v sta pi\v se)}.\\
Mi \' cemo se baviti ovim dijelom koji lokalno pretra\v zuje prostor rje\v senja i daje nam jedno rije\v senje kao rezultat.

\section{slajd}
Algoritmi koje smo odabrali spadaju u klasu populacijskih algoritama, posebno u klasu algoritama baziranih na rojevim \v cestica. Ova vrsta algoritama se oslanja na kolektivno pona\v sanje samostalnih \v cestica u roju. Primjer su mravlji i p\v celinji algoritam.

\section{slajd}
Sada \' cemo dati kratki pregled znasvene i stru\v cne literature u tom podru\v cju unazad par godina.\\
\emph{Pro\v citat godinu, koji je algoritam u igri, mo\v zda pokoju funkciju}

\section{slajd}
\emph{Pro\v citat godinu, koji je algoritam u igri, mo\v zda pokoju funkciju}

\section{slajd}
\emph{Pro\v citat godinu, koji je algoritam u igri, mo\v zda pokoju funkciju}
\section{slajd}

\emph{Pro\v citat godinu, koji je algoritam u igri, mo\v zda pokoju funkciju}
\section{slajd}

\emph{Pro\v citat godinu, koji je algoritam u igri, mo\v zda pokoju funkciju}\\
Ovaj \v clanak je ustvari bio i motivacija za ovaj projekt, i vidimo da je nedavno izdan te da je tema jako popularna u znastvenoj zajednici.

\section{slajd}
No, \v sto je zaista bio na\v s projektni zadatak?\\
Mi smo odabrali 2 algoritma koja se baziraju na roju \v cestivca, PSO i ABC o kojima \v cemo malo kasnije, te smo osmisli novi algoritam koji je mini hibrid ta dva.\\
Za prikaz rezultata na\v seg projekta i interakciju s korisnikom osmislili smo i izradili grafi\v cko web su\v celje.

\section{slajd}
Evo kratkog pregleda kori\v stenih algoritama sa kratkim opisom na\v cina kako oni rade.\\
Prvi lagoritam je p\v celinji algoritam (dalje \v cemo ga zvati ABC). Njegove mane su veliki broj parametara koje koristi \emph{(pokazat rukom)} i to \v sto ne konvergira. Prednost mu je \v sto po samoj prirodi svog rada ima neku vrstu lokalne pretrage a veliki brj \v cestica koje sudjeluju u izra\v cunu pove\' cavaju \v sansu pronalska povoljnog rezultata.

\section{slajd}
\emph{pro\v citaj slajd, ma\v si rukama dok obja\v snjavas, crtaj malo po plo\v ci}

\section{slajd}
Zatim, implementirali smo i PSO\\
\emph{pro\v citaj slajd to\v cke 2 i 3}

\section{slajd}
\emph{pro\v citaj slajd, ma\v si rukama dok obja\v snjavas, crtaj malo po plo\v ci}

\section{slajd}
Ovo je algoritam koji smo mi osmislili. Baziran je na PSO, ali smo mu dodali lokalnu pretragu motiviranu lokalnom pretragom iz ABC. Razlika je \v sto na\v sa lokalna pretraga prtera\v zuje 'orbite' najbolje \v cestice te time pronalazimo najoptimalniji smijer za budu\' ce kretanje same \v cestice i cijelog roja.


\section{slajd}
\emph{pro\v citaj slajd, ma\v si rukama dok obja\v snjavas, crtaj malo po plo\v ci, skrenut pa\v znju na *}

\section{slajd}

\end{document}
